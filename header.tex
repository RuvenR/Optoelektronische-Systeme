\documentclass[12pt,a4paper,oneside,ngerman,hidelinks]{article}

\usepackage[a4paper, left= 3.5cm, right=2.5cm, top=2.5cm, bottom=2.5cm]{geometry} %Festlegen der Ränder
\setlength{\headheight}{12.5pt} %Ändern des Zugewiesenen Platzes für Header


\usepackage{setspace} %Ändern Zeilenabstand
\setstretch{1.433}
\usepackage{csquotes} %Zitieren für BibLaTeX
\usepackage[T1]{fontenc} %Fonts mit europäischer Codierung

\usepackage{fontspec} %Festlegen einer Custom Fost aus Systemfonts (XeLaTeX)
\usepackage{polyglossia} %Alternative für babel (XeLaTeX)
\setmainlanguage{german} %für fontspec
\setotherlanguages{english}
\setmainfont{Times New Roman} %für fontspec
\usepackage{placeins} %Verhindern, dass sich floats über Punkt hinaus bewegen
\usepackage{caption} %Mehrzeilige Captions werden mit Einzug formatiert
\captionsetup{format=hang}
\usepackage{siunitx}
\sisetup{locale = DE}

\usepackage{amsmath} %Anwendung von mathematischen Funktionen
\usepackage{amsfonts} %u. a. mathematische Symbole/Fonts
\usepackage{amssymb} %u. a. mathematische Symbole/Fonts
\usepackage{graphicx} %Einbinden von Grafiken
\usepackage{tabulary} %verbesserte Tabellen
\usepackage{longtable} %Tabellen mit Seitenumbruch
\usepackage[dvipsnames,table]{xcolor} % Färben von Tabelle
\definecolor{light-gray}{HTML}{E5E4E2}
\definecolor{light-red}{HTML}{ffcccc}
\usepackage{multirow} %mehrzeilige Tabellenzellen
\usepackage{pgfplots} %Plotten von Werten
\pgfplotsset{compat=1.17}
\usepackage{acro}
% https://tex.stackexchange.com/questions/624047/acro-package-move-translation-to-the-next-line-with-long-table, 12.01.2022
\DeclareAcroProperty?{foreign-newline}
\NewAcroTemplate[list]{LongtableForeign}
{
    \AcroNeedPackage {array,longtable}
    \acronymsmapF
    {
        \AcroAddRow
        {
            \acrowrite {short}
            \acroifT {alt} { / } \acrowrite {alt}
            &
            \acrowrite {list}
            \acroifbooleanTF{foreign-newline}{
                \newline
            \itshape \acroifT{foreign}{(}\acrowrite{foreign}\acroifT{foreign}{)}}
            {\itshape \acroifT{foreign}{(}\acrowrite{foreign}\acroifT{foreign}{)}}  
            \acropagefill
            \acropages
            {\acrotranslate {page} \nobreakspace }
            {\acrotranslate {pages} \nobreakspace }
            \tabularnewline
        }
    }
    { \AcroRerun }  
    \acroheading
    \acropreamble
    \par \noindent
    \begin {longtable} {>{\bfseries}lp{.7\linewidth}}
    \AcronymTable
    \end {longtable}
}
%---
\usepackage{titletoc}


\setlength{\parindent}{0pt} %Zeilen nach neuem Absatz 0px einrücken
\usepackage[hang,flushmargin]{footmisc} %Einzug Fußnote entfernen
\addto\captionsgerman{\renewcommand{\figurename}{Abb.}}%Abkürzen der Bezeichnung für Abbildung
\setlength{\abovecaptionskip}{3pt plus 3pt minus 2pt} %Festlegen von Platz über Bildunterschrift
\usepackage{wrapfig} %Bilder von Text umfließen lassen
\usepackage{subcaption} %subfigures
\usepackage{geometry} %Vereinfachung des Seitenlayouts
\usepackage{setspace} %flexible Änderung von Zeilenabständen
\usepackage{scalerel}

\newcounter{alteSeitenzahl} %Zähler für Umschalten zwischen römischen und arabischen Zahlen
\usepackage{chngcntr} %Änderung von Countern wird ermöglicht

\newcounter{appendno} % Counter für Anhang
\newcommand{\stepappend}{%
	\stepcounter{appendno}%
	}
\newcommand{\rappend}[1]{\refstepcounter{appendno}\label{#1}}



\usepackage[titles]{tocloft} % in Abbildungs- /Tabellenverzeichnis Bezeichnung vor Zahl
\renewcommand{\cfttabpresnum}{Tab. }
\renewcommand{\cftfigpresnum}{Abb. }
\settowidth{\cfttabnumwidth}{Abb. 10\quad}
\settowidth{\cftfignumwidth}{Abb. 10\quad}


\AtBeginDocument{
\addtolength{\abovedisplayskip}{-2ex}
\addtolength{\abovedisplayshortskip}{-2ex}
\addtolength{\belowdisplayskip}{-0.5ex}
\addtolength{\belowdisplayshortskip}{-0.5ex}
}

\usepackage[headsepline, markcase=ignoreuppercase]{scrlayer-scrpage} %Definition von Seitenlayout
\automark{subsection} %Text für Titelseite
\renewcommand{\headfont}{\normalfont\rmfamily}

\usepackage{titlesec}
\graphicspath{{Bilder/}}

\usepackage[citestyle=numeric,bibstyle=ieee,doi=false,isbn=false,url=true,eprint=false, defernumbers=true, sorting=none,minnames=1,maxnames=99]{biblatex}

\DefineBibliographyStrings{german}{andothers={et\addabbrvspace al\adddot},jourvol={Bd\adddot}}
\addbibresource{referenzen.bib}

\newcommand*\oldurlbreaks{}
\let\oldurlbreaks=\UrlBreaks
\usepackage{tocbibind}
\PassOptionsToPackage{hyphens}{url}\usepackage[pdfauthor={Ruven Rüger},pdftitle={Bachelorarbeit Ruven Rüger}]{hyperref} %Verlinken von Überschriften (muss letztes Package sein!)
\expandafter\def\expandafter\UrlBreaks\expandafter{\UrlBreaks
  \do\a\do\b\do\c\do\d\do\e\do\f\do\g\do\h\do\i\do\j\do\k\do\l\do\m\do\n\do\o\do\p\do\q\do\r\do\s\do\t\do\u\do\v\do\w\do\x\do\y\do\z\do\A\do\B\do\C\do\D\do\E\do\F\do\G\do\H\do\I\do\J\do\K\do\L\do\M\do\N\do\O\do\P\do\Q\do\R\do\S\do\T\do\U\do\V\do\W\do\X\do\Y\do\Z}