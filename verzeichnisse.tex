\pagenumbering{Roman}
\newgeometry{left=2.5cm,bottom=1cm,top=0.7cm,right=2.5cm}
    \noindent
	\thispagestyle{empty}
	\pagenumbering{Roman}
	\vspace*{1.5mm}
	\begin{center}
 		\makebox[\textwidth]{\includegraphics[width=1.1\textwidth]{EAH_header}}
	\end{center}
	\begin{center}
		\vspace*{20mm}
        \large
        Fachbereich Elektrotechnik und Informationstechnik\\
        \vspace*{12mm}
        \huge
        \textbf{SEMINARARBEIT}\\
        \large
        \vspace*{14.5mm}
        %\textbf{zur Erlangung des akademischen Grades}\\
        %Bachelor of Engineering (B. Eng.)
        \Large
        \begin{spacing}{1.3}
        \textbf{RaioSat -- Erstellung eines 3D-Modells der Beobachtungssituation}
        \end{spacing}
        \vspace{39mm} 
\begin{spacing}{1.8}
\begin{large}
\begin{tabular}{ l l } 
 Eingereicht von: & Kreusch, Richard \hspace{4mm}(641294) \\
 & Rüger, Ruven\hspace{12.5mm}(644794)\\
 %Geboren am: & 29.08.2000 in Kulmbach\\
 %Matrikelnummer: & 644794 \\
 Studiengang: & Master Elektrotechnik/Informationstechnik \\
 Modul:& Optoelektronische Systeme \\
 Betreuer: & Prof. Dr. Alexander Richter\\
 %Mentor (Jena-Optronik GmbH): & M. Eng. Fabian Lami\\
 %Datum der Themenausgabe: & 22.02.2022\\
 Datum der Abgabe: & 01.01.1970\\

\end{tabular}
\end{large}
\end{spacing}      
\end{center}
\vspace{3mm}
\begin{center}
	\makebox[\textwidth]{\includegraphics[width=1.1\textwidth]{EAH_footer}}
\end{center}
\restoregeometry
\newpage


\renewcommand{\cftsecleader}{\cftdotfill{\cftdotsep}}
%\addcontentsline{toc}{section}{Inhaltsverzeichnis}
\markright{Inhaltsverzeichnis}
\section*{Inhaltsverzeichnis}
\printcontents{ }{3}{}
\newpage
\listoffigures
\newpage
\listoftables
\automark{section}
\addcontentsline{toc}{section}{Abkürzungsverzeichnis}
\markright{Abkürzungsverzeichnis}
\vspace{1cm}
\section*{Abkürzungsverzeichnis}
\acuseall
\printacronyms[display=all,heading=none]
\newpage

\addcontentsline{toc}{section}{Symbolverzeichnis}
\markright{Symbolverzeichnis}
\section*{Symbolverzeichnis}
\begin{center}
		\begin{longtable}{p{0.62\textwidth} p{0.15\textwidth} p{0.15\textwidth}}
		\hline
		Bezeichnung  & Symbol &  Einheit \\
		\hline
		\endhead
		\label{tab:ergBerMess}
		\hspace{-1mm}Ausgangsspannung DAC & $U_{D\!AC}$ & $V$\\
		Ausgangsspannung CTIA & $u_{a}$ & $V$\\
		 
	\end{longtable}
\end{center}
\clearpage

\addcontentsline{toc}{section}{Abstract}
\markright{Abstract}
\section*{Abstract}

\clearpage